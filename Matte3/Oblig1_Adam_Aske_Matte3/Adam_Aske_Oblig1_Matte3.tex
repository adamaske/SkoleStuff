\documentclass[a4paper,norsk]{article}
\usepackage[utf8]{inputenc}
\usepackage[T1]{fontenc,url}
\usepackage{babel,textcomp}
\usepackage{graphicx}
\usepackage{amsmath}
\usepackage{cleveref}
\usepackage[cmyk]{xcolor}
\usepackage{listings}

\lstset {language=C++,    
backgroundcolor=\color{yellow!20},    
commentstyle=\color{green},    
%keywordstyle=\color{blue},    
basicstyle=\footnotesize}
\urlstyle{sf}
\title{Oblig 1 Matte 3}
\date{\today}
\author{Adam Aske}
\newpage
\begin{document}
\maketitle
\tableofcontents
\newpage
 \section{Github}
Link til min branch : https://github.com/Hedmark-University-College-SPIM/3Dprog22/tree/AdamA
\section{Del 1}
Jeg har valgt funksjonen; f(x, y) = sin(PI*x)*sin(PI*y). Omerådet 0 < x < 1, 0 < y < 3 og steg = 0.2.
Funkjsonen tar inn en array og en størrelse. Først blir arrayen fylt med tilfeldige tall. 
\begin{lstlisting}[language=C++, caption={trianglesurface.cpp}]
//Create triangle
float xmin=0.0f, xmax=1.0f, ymin=0.0f, ymax=1.0f, h=0.1f;
for (auto x=xmin; x<xmax; x+=h)
{
	for (auto y=ymin; y<ymax; y+=h)
	{
              float z = func(x, y);
              mVertices.push_back(Vertex{x,y,z,x,y,z});
              z = func(x+h, y);
              mVertices.push_back(Vertex{x+h,y,z,x,y,z});
              z = func(x, y+h);
              mVertices.push_back(Vertex{x,y+h,z,x,y,z});
              mVertices.push_back(Vertex{x,y+h,z,x,y,z});
              z = func(x+h, y);
              mVertices.push_back(Vertex{x+h,y,z,x,y,z});
              z = func(x+h, y+h);
              mVertices.push_back(Vertex{x+h,y+h,z,x,y,z});
	}
}
\end{lstlisting}
\begin{lstlisting}[language=C++, caption={trianglesurface.hh}]
static float func(float x, float y){
       //Matte oblig funksjon
       return pow(x, 3) * y;
   }
\end{lstlisting}
\section{Lese og skrive til fil}
\begin{lstlisting}[language=C++, caption={trianglesurface.cpp}]
void TriangleSurface::readFile(std::string fileName) {
    	std::ifstream inn;
       inn.open(fileName.c_str());
       if (inn.is_open())
       {
        	int n;
             Vertex vertex;
             inn >> n;
             mVertices.reserve(n);
             for (int i=0; i<n; i++) {
             	inn >> vertex;
                   mVertices.push_back(vertex);
              }
              inn.close();
          }
}

void TriangleSurface::writeFile(std::string fileName){
	std::ofstream wF;
       wF.open(fileName.c_str())
       if(wF.is_open())
       {
        	wF << mVertices.size() << "\n";
            for (int i = 0; i < mVertices.size(); i++)
            {
                wF << mVertices[i] << "\n";
            }
        }
        else
        {
            std::cout << "Failed to write to file.\n";
        }
        wF.close();
}
\end{lstlisting}

\section{Del 2}
\section{A}
Analytisk utregning for volumet av funksjonen.
\(\int_{0}^{1} \int_{0}^{1} x^3 * y \,dy, dx\)
\newline
\(\int_{0}^{1} x^3*y \,dy\) = \(x^3\int y \,dy\) = \(x^3 * (y^2/2) \) = \( x^3y^2/2  \) = \( x^3*1^2/2\) = \(x^3/2\)
\newline \(\int x^3/2\) = \(1/2 \int x^3\,dx\) = \(1/2* x^4/4\) = \( x^4/8 \)
\newline\(\int_{0}^{1} \int_{0}^{1} x^3 * y \,dy, dx\) = 1/2

\section{B)}
For å regne integralet numerisk lagde jeg en funksjon i trianglesurface.cpp og skriver resultatene til en fil.
Funksjoner gjør det 4 ganger og halverer steg lengden for hver iterasjon. Resultatene blir lagret i Numerisk.txt.
\begin{lstlisting}[language=C++, caption={trianglesurface.cpp}]
void TriangleSurface::CalculateNumerical(){
    	std::ofstream file;
   	file.open("Numerisk.txt");
	if(file.is_open())
       {
        	float xmin= 0.0f, xmax = 1.0f, ymin = 0.0f, ymax  = 1.0f,h = 0.1f, result = 0;
             for(int i = 0; i < 4; i++)
             {
                for(auto x = xmin; x < xmax; x+=h)
                {
                    for(auto y = ymin; y < ymax; y+= h)
                    {
                        float z = func(x, y) * pow(h, 2);
                        result += z;
                    }
                }
            h = h / 2;
            file << result << "\n";
            }
        }
        else
        {
            std::cout << "Failed to write to file.\n";
        }
        file.close();
}
\end{lstlisting}
Resultatene ble:
h1 = 0.091125,
h2 = 0.198297,
h3 = 0.332908,
h 4 =0.462652


\section{Resultat}
Den numeriske utregningen går nærmere og nærmere svaret jeg fikk fra manuell utergning; 1/2. 

\end{document}